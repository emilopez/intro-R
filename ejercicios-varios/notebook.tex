
% Default to the notebook output style

    


% Inherit from the specified cell style.




    
\documentclass[11pt]{article}

    
    
    \usepackage[T1]{fontenc}
    % Nicer default font (+ math font) than Computer Modern for most use cases
    \usepackage{mathpazo}

    % Basic figure setup, for now with no caption control since it's done
    % automatically by Pandoc (which extracts ![](path) syntax from Markdown).
    \usepackage{graphicx}
    % We will generate all images so they have a width \maxwidth. This means
    % that they will get their normal width if they fit onto the page, but
    % are scaled down if they would overflow the margins.
    \makeatletter
    \def\maxwidth{\ifdim\Gin@nat@width>\linewidth\linewidth
    \else\Gin@nat@width\fi}
    \makeatother
    \let\Oldincludegraphics\includegraphics
    % Set max figure width to be 80% of text width, for now hardcoded.
    \renewcommand{\includegraphics}[1]{\Oldincludegraphics[width=.8\maxwidth]{#1}}
    % Ensure that by default, figures have no caption (until we provide a
    % proper Figure object with a Caption API and a way to capture that
    % in the conversion process - todo).
    \usepackage{caption}
    \DeclareCaptionLabelFormat{nolabel}{}
    \captionsetup{labelformat=nolabel}

    \usepackage{adjustbox} % Used to constrain images to a maximum size 
    \usepackage{xcolor} % Allow colors to be defined
    \usepackage{enumerate} % Needed for markdown enumerations to work
    \usepackage{geometry} % Used to adjust the document margins
    \usepackage{amsmath} % Equations
    \usepackage{amssymb} % Equations
    \usepackage{textcomp} % defines textquotesingle
    % Hack from http://tex.stackexchange.com/a/47451/13684:
    \AtBeginDocument{%
        \def\PYZsq{\textquotesingle}% Upright quotes in Pygmentized code
    }
    \usepackage{upquote} % Upright quotes for verbatim code
    \usepackage{eurosym} % defines \euro
    \usepackage[mathletters]{ucs} % Extended unicode (utf-8) support
    \usepackage[utf8x]{inputenc} % Allow utf-8 characters in the tex document
    \usepackage{fancyvrb} % verbatim replacement that allows latex
    \usepackage{grffile} % extends the file name processing of package graphics 
                         % to support a larger range 
    % The hyperref package gives us a pdf with properly built
    % internal navigation ('pdf bookmarks' for the table of contents,
    % internal cross-reference links, web links for URLs, etc.)
    \usepackage{hyperref}
    \usepackage{longtable} % longtable support required by pandoc >1.10
    \usepackage{booktabs}  % table support for pandoc > 1.12.2
    \usepackage[inline]{enumitem} % IRkernel/repr support (it uses the enumerate* environment)
    \usepackage[normalem]{ulem} % ulem is needed to support strikethroughs (\sout)
                                % normalem makes italics be italics, not underlines
    

    
    
    % Colors for the hyperref package
    \definecolor{urlcolor}{rgb}{0,.145,.698}
    \definecolor{linkcolor}{rgb}{.71,0.21,0.01}
    \definecolor{citecolor}{rgb}{.12,.54,.11}

    % ANSI colors
    \definecolor{ansi-black}{HTML}{3E424D}
    \definecolor{ansi-black-intense}{HTML}{282C36}
    \definecolor{ansi-red}{HTML}{E75C58}
    \definecolor{ansi-red-intense}{HTML}{B22B31}
    \definecolor{ansi-green}{HTML}{00A250}
    \definecolor{ansi-green-intense}{HTML}{007427}
    \definecolor{ansi-yellow}{HTML}{DDB62B}
    \definecolor{ansi-yellow-intense}{HTML}{B27D12}
    \definecolor{ansi-blue}{HTML}{208FFB}
    \definecolor{ansi-blue-intense}{HTML}{0065CA}
    \definecolor{ansi-magenta}{HTML}{D160C4}
    \definecolor{ansi-magenta-intense}{HTML}{A03196}
    \definecolor{ansi-cyan}{HTML}{60C6C8}
    \definecolor{ansi-cyan-intense}{HTML}{258F8F}
    \definecolor{ansi-white}{HTML}{C5C1B4}
    \definecolor{ansi-white-intense}{HTML}{A1A6B2}

    % commands and environments needed by pandoc snippets
    % extracted from the output of `pandoc -s`
    \providecommand{\tightlist}{%
      \setlength{\itemsep}{0pt}\setlength{\parskip}{0pt}}
    \DefineVerbatimEnvironment{Highlighting}{Verbatim}{commandchars=\\\{\}}
    % Add ',fontsize=\small' for more characters per line
    \newenvironment{Shaded}{}{}
    \newcommand{\KeywordTok}[1]{\textcolor[rgb]{0.00,0.44,0.13}{\textbf{{#1}}}}
    \newcommand{\DataTypeTok}[1]{\textcolor[rgb]{0.56,0.13,0.00}{{#1}}}
    \newcommand{\DecValTok}[1]{\textcolor[rgb]{0.25,0.63,0.44}{{#1}}}
    \newcommand{\BaseNTok}[1]{\textcolor[rgb]{0.25,0.63,0.44}{{#1}}}
    \newcommand{\FloatTok}[1]{\textcolor[rgb]{0.25,0.63,0.44}{{#1}}}
    \newcommand{\CharTok}[1]{\textcolor[rgb]{0.25,0.44,0.63}{{#1}}}
    \newcommand{\StringTok}[1]{\textcolor[rgb]{0.25,0.44,0.63}{{#1}}}
    \newcommand{\CommentTok}[1]{\textcolor[rgb]{0.38,0.63,0.69}{\textit{{#1}}}}
    \newcommand{\OtherTok}[1]{\textcolor[rgb]{0.00,0.44,0.13}{{#1}}}
    \newcommand{\AlertTok}[1]{\textcolor[rgb]{1.00,0.00,0.00}{\textbf{{#1}}}}
    \newcommand{\FunctionTok}[1]{\textcolor[rgb]{0.02,0.16,0.49}{{#1}}}
    \newcommand{\RegionMarkerTok}[1]{{#1}}
    \newcommand{\ErrorTok}[1]{\textcolor[rgb]{1.00,0.00,0.00}{\textbf{{#1}}}}
    \newcommand{\NormalTok}[1]{{#1}}
    
    % Additional commands for more recent versions of Pandoc
    \newcommand{\ConstantTok}[1]{\textcolor[rgb]{0.53,0.00,0.00}{{#1}}}
    \newcommand{\SpecialCharTok}[1]{\textcolor[rgb]{0.25,0.44,0.63}{{#1}}}
    \newcommand{\VerbatimStringTok}[1]{\textcolor[rgb]{0.25,0.44,0.63}{{#1}}}
    \newcommand{\SpecialStringTok}[1]{\textcolor[rgb]{0.73,0.40,0.53}{{#1}}}
    \newcommand{\ImportTok}[1]{{#1}}
    \newcommand{\DocumentationTok}[1]{\textcolor[rgb]{0.73,0.13,0.13}{\textit{{#1}}}}
    \newcommand{\AnnotationTok}[1]{\textcolor[rgb]{0.38,0.63,0.69}{\textbf{\textit{{#1}}}}}
    \newcommand{\CommentVarTok}[1]{\textcolor[rgb]{0.38,0.63,0.69}{\textbf{\textit{{#1}}}}}
    \newcommand{\VariableTok}[1]{\textcolor[rgb]{0.10,0.09,0.49}{{#1}}}
    \newcommand{\ControlFlowTok}[1]{\textcolor[rgb]{0.00,0.44,0.13}{\textbf{{#1}}}}
    \newcommand{\OperatorTok}[1]{\textcolor[rgb]{0.40,0.40,0.40}{{#1}}}
    \newcommand{\BuiltInTok}[1]{{#1}}
    \newcommand{\ExtensionTok}[1]{{#1}}
    \newcommand{\PreprocessorTok}[1]{\textcolor[rgb]{0.74,0.48,0.00}{{#1}}}
    \newcommand{\AttributeTok}[1]{\textcolor[rgb]{0.49,0.56,0.16}{{#1}}}
    \newcommand{\InformationTok}[1]{\textcolor[rgb]{0.38,0.63,0.69}{\textbf{\textit{{#1}}}}}
    \newcommand{\WarningTok}[1]{\textcolor[rgb]{0.38,0.63,0.69}{\textbf{\textit{{#1}}}}}
    
    
    % Define a nice break command that doesn't care if a line doesn't already
    % exist.
    \def\br{\hspace*{\fill} \\* }
    % Math Jax compatability definitions
    \def\gt{>}
    \def\lt{<}
    % Document parameters
    \title{Efectos\_Mixtos\_Ej5\_ejercicios\_mixtospdf}
    
    
    

    % Pygments definitions
    
\makeatletter
\def\PY@reset{\let\PY@it=\relax \let\PY@bf=\relax%
    \let\PY@ul=\relax \let\PY@tc=\relax%
    \let\PY@bc=\relax \let\PY@ff=\relax}
\def\PY@tok#1{\csname PY@tok@#1\endcsname}
\def\PY@toks#1+{\ifx\relax#1\empty\else%
    \PY@tok{#1}\expandafter\PY@toks\fi}
\def\PY@do#1{\PY@bc{\PY@tc{\PY@ul{%
    \PY@it{\PY@bf{\PY@ff{#1}}}}}}}
\def\PY#1#2{\PY@reset\PY@toks#1+\relax+\PY@do{#2}}

\expandafter\def\csname PY@tok@w\endcsname{\def\PY@tc##1{\textcolor[rgb]{0.73,0.73,0.73}{##1}}}
\expandafter\def\csname PY@tok@c\endcsname{\let\PY@it=\textit\def\PY@tc##1{\textcolor[rgb]{0.25,0.50,0.50}{##1}}}
\expandafter\def\csname PY@tok@cp\endcsname{\def\PY@tc##1{\textcolor[rgb]{0.74,0.48,0.00}{##1}}}
\expandafter\def\csname PY@tok@k\endcsname{\let\PY@bf=\textbf\def\PY@tc##1{\textcolor[rgb]{0.00,0.50,0.00}{##1}}}
\expandafter\def\csname PY@tok@kp\endcsname{\def\PY@tc##1{\textcolor[rgb]{0.00,0.50,0.00}{##1}}}
\expandafter\def\csname PY@tok@kt\endcsname{\def\PY@tc##1{\textcolor[rgb]{0.69,0.00,0.25}{##1}}}
\expandafter\def\csname PY@tok@o\endcsname{\def\PY@tc##1{\textcolor[rgb]{0.40,0.40,0.40}{##1}}}
\expandafter\def\csname PY@tok@ow\endcsname{\let\PY@bf=\textbf\def\PY@tc##1{\textcolor[rgb]{0.67,0.13,1.00}{##1}}}
\expandafter\def\csname PY@tok@nb\endcsname{\def\PY@tc##1{\textcolor[rgb]{0.00,0.50,0.00}{##1}}}
\expandafter\def\csname PY@tok@nf\endcsname{\def\PY@tc##1{\textcolor[rgb]{0.00,0.00,1.00}{##1}}}
\expandafter\def\csname PY@tok@nc\endcsname{\let\PY@bf=\textbf\def\PY@tc##1{\textcolor[rgb]{0.00,0.00,1.00}{##1}}}
\expandafter\def\csname PY@tok@nn\endcsname{\let\PY@bf=\textbf\def\PY@tc##1{\textcolor[rgb]{0.00,0.00,1.00}{##1}}}
\expandafter\def\csname PY@tok@ne\endcsname{\let\PY@bf=\textbf\def\PY@tc##1{\textcolor[rgb]{0.82,0.25,0.23}{##1}}}
\expandafter\def\csname PY@tok@nv\endcsname{\def\PY@tc##1{\textcolor[rgb]{0.10,0.09,0.49}{##1}}}
\expandafter\def\csname PY@tok@no\endcsname{\def\PY@tc##1{\textcolor[rgb]{0.53,0.00,0.00}{##1}}}
\expandafter\def\csname PY@tok@nl\endcsname{\def\PY@tc##1{\textcolor[rgb]{0.63,0.63,0.00}{##1}}}
\expandafter\def\csname PY@tok@ni\endcsname{\let\PY@bf=\textbf\def\PY@tc##1{\textcolor[rgb]{0.60,0.60,0.60}{##1}}}
\expandafter\def\csname PY@tok@na\endcsname{\def\PY@tc##1{\textcolor[rgb]{0.49,0.56,0.16}{##1}}}
\expandafter\def\csname PY@tok@nt\endcsname{\let\PY@bf=\textbf\def\PY@tc##1{\textcolor[rgb]{0.00,0.50,0.00}{##1}}}
\expandafter\def\csname PY@tok@nd\endcsname{\def\PY@tc##1{\textcolor[rgb]{0.67,0.13,1.00}{##1}}}
\expandafter\def\csname PY@tok@s\endcsname{\def\PY@tc##1{\textcolor[rgb]{0.73,0.13,0.13}{##1}}}
\expandafter\def\csname PY@tok@sd\endcsname{\let\PY@it=\textit\def\PY@tc##1{\textcolor[rgb]{0.73,0.13,0.13}{##1}}}
\expandafter\def\csname PY@tok@si\endcsname{\let\PY@bf=\textbf\def\PY@tc##1{\textcolor[rgb]{0.73,0.40,0.53}{##1}}}
\expandafter\def\csname PY@tok@se\endcsname{\let\PY@bf=\textbf\def\PY@tc##1{\textcolor[rgb]{0.73,0.40,0.13}{##1}}}
\expandafter\def\csname PY@tok@sr\endcsname{\def\PY@tc##1{\textcolor[rgb]{0.73,0.40,0.53}{##1}}}
\expandafter\def\csname PY@tok@ss\endcsname{\def\PY@tc##1{\textcolor[rgb]{0.10,0.09,0.49}{##1}}}
\expandafter\def\csname PY@tok@sx\endcsname{\def\PY@tc##1{\textcolor[rgb]{0.00,0.50,0.00}{##1}}}
\expandafter\def\csname PY@tok@m\endcsname{\def\PY@tc##1{\textcolor[rgb]{0.40,0.40,0.40}{##1}}}
\expandafter\def\csname PY@tok@gh\endcsname{\let\PY@bf=\textbf\def\PY@tc##1{\textcolor[rgb]{0.00,0.00,0.50}{##1}}}
\expandafter\def\csname PY@tok@gu\endcsname{\let\PY@bf=\textbf\def\PY@tc##1{\textcolor[rgb]{0.50,0.00,0.50}{##1}}}
\expandafter\def\csname PY@tok@gd\endcsname{\def\PY@tc##1{\textcolor[rgb]{0.63,0.00,0.00}{##1}}}
\expandafter\def\csname PY@tok@gi\endcsname{\def\PY@tc##1{\textcolor[rgb]{0.00,0.63,0.00}{##1}}}
\expandafter\def\csname PY@tok@gr\endcsname{\def\PY@tc##1{\textcolor[rgb]{1.00,0.00,0.00}{##1}}}
\expandafter\def\csname PY@tok@ge\endcsname{\let\PY@it=\textit}
\expandafter\def\csname PY@tok@gs\endcsname{\let\PY@bf=\textbf}
\expandafter\def\csname PY@tok@gp\endcsname{\let\PY@bf=\textbf\def\PY@tc##1{\textcolor[rgb]{0.00,0.00,0.50}{##1}}}
\expandafter\def\csname PY@tok@go\endcsname{\def\PY@tc##1{\textcolor[rgb]{0.53,0.53,0.53}{##1}}}
\expandafter\def\csname PY@tok@gt\endcsname{\def\PY@tc##1{\textcolor[rgb]{0.00,0.27,0.87}{##1}}}
\expandafter\def\csname PY@tok@err\endcsname{\def\PY@bc##1{\setlength{\fboxsep}{0pt}\fcolorbox[rgb]{1.00,0.00,0.00}{1,1,1}{\strut ##1}}}
\expandafter\def\csname PY@tok@kc\endcsname{\let\PY@bf=\textbf\def\PY@tc##1{\textcolor[rgb]{0.00,0.50,0.00}{##1}}}
\expandafter\def\csname PY@tok@kd\endcsname{\let\PY@bf=\textbf\def\PY@tc##1{\textcolor[rgb]{0.00,0.50,0.00}{##1}}}
\expandafter\def\csname PY@tok@kn\endcsname{\let\PY@bf=\textbf\def\PY@tc##1{\textcolor[rgb]{0.00,0.50,0.00}{##1}}}
\expandafter\def\csname PY@tok@kr\endcsname{\let\PY@bf=\textbf\def\PY@tc##1{\textcolor[rgb]{0.00,0.50,0.00}{##1}}}
\expandafter\def\csname PY@tok@bp\endcsname{\def\PY@tc##1{\textcolor[rgb]{0.00,0.50,0.00}{##1}}}
\expandafter\def\csname PY@tok@fm\endcsname{\def\PY@tc##1{\textcolor[rgb]{0.00,0.00,1.00}{##1}}}
\expandafter\def\csname PY@tok@vc\endcsname{\def\PY@tc##1{\textcolor[rgb]{0.10,0.09,0.49}{##1}}}
\expandafter\def\csname PY@tok@vg\endcsname{\def\PY@tc##1{\textcolor[rgb]{0.10,0.09,0.49}{##1}}}
\expandafter\def\csname PY@tok@vi\endcsname{\def\PY@tc##1{\textcolor[rgb]{0.10,0.09,0.49}{##1}}}
\expandafter\def\csname PY@tok@vm\endcsname{\def\PY@tc##1{\textcolor[rgb]{0.10,0.09,0.49}{##1}}}
\expandafter\def\csname PY@tok@sa\endcsname{\def\PY@tc##1{\textcolor[rgb]{0.73,0.13,0.13}{##1}}}
\expandafter\def\csname PY@tok@sb\endcsname{\def\PY@tc##1{\textcolor[rgb]{0.73,0.13,0.13}{##1}}}
\expandafter\def\csname PY@tok@sc\endcsname{\def\PY@tc##1{\textcolor[rgb]{0.73,0.13,0.13}{##1}}}
\expandafter\def\csname PY@tok@dl\endcsname{\def\PY@tc##1{\textcolor[rgb]{0.73,0.13,0.13}{##1}}}
\expandafter\def\csname PY@tok@s2\endcsname{\def\PY@tc##1{\textcolor[rgb]{0.73,0.13,0.13}{##1}}}
\expandafter\def\csname PY@tok@sh\endcsname{\def\PY@tc##1{\textcolor[rgb]{0.73,0.13,0.13}{##1}}}
\expandafter\def\csname PY@tok@s1\endcsname{\def\PY@tc##1{\textcolor[rgb]{0.73,0.13,0.13}{##1}}}
\expandafter\def\csname PY@tok@mb\endcsname{\def\PY@tc##1{\textcolor[rgb]{0.40,0.40,0.40}{##1}}}
\expandafter\def\csname PY@tok@mf\endcsname{\def\PY@tc##1{\textcolor[rgb]{0.40,0.40,0.40}{##1}}}
\expandafter\def\csname PY@tok@mh\endcsname{\def\PY@tc##1{\textcolor[rgb]{0.40,0.40,0.40}{##1}}}
\expandafter\def\csname PY@tok@mi\endcsname{\def\PY@tc##1{\textcolor[rgb]{0.40,0.40,0.40}{##1}}}
\expandafter\def\csname PY@tok@il\endcsname{\def\PY@tc##1{\textcolor[rgb]{0.40,0.40,0.40}{##1}}}
\expandafter\def\csname PY@tok@mo\endcsname{\def\PY@tc##1{\textcolor[rgb]{0.40,0.40,0.40}{##1}}}
\expandafter\def\csname PY@tok@ch\endcsname{\let\PY@it=\textit\def\PY@tc##1{\textcolor[rgb]{0.25,0.50,0.50}{##1}}}
\expandafter\def\csname PY@tok@cm\endcsname{\let\PY@it=\textit\def\PY@tc##1{\textcolor[rgb]{0.25,0.50,0.50}{##1}}}
\expandafter\def\csname PY@tok@cpf\endcsname{\let\PY@it=\textit\def\PY@tc##1{\textcolor[rgb]{0.25,0.50,0.50}{##1}}}
\expandafter\def\csname PY@tok@c1\endcsname{\let\PY@it=\textit\def\PY@tc##1{\textcolor[rgb]{0.25,0.50,0.50}{##1}}}
\expandafter\def\csname PY@tok@cs\endcsname{\let\PY@it=\textit\def\PY@tc##1{\textcolor[rgb]{0.25,0.50,0.50}{##1}}}

\def\PYZbs{\char`\\}
\def\PYZus{\char`\_}
\def\PYZob{\char`\{}
\def\PYZcb{\char`\}}
\def\PYZca{\char`\^}
\def\PYZam{\char`\&}
\def\PYZlt{\char`\<}
\def\PYZgt{\char`\>}
\def\PYZsh{\char`\#}
\def\PYZpc{\char`\%}
\def\PYZdl{\char`\$}
\def\PYZhy{\char`\-}
\def\PYZsq{\char`\'}
\def\PYZdq{\char`\"}
\def\PYZti{\char`\~}
% for compatibility with earlier versions
\def\PYZat{@}
\def\PYZlb{[}
\def\PYZrb{]}
\makeatother


    % Exact colors from NB
    \definecolor{incolor}{rgb}{0.0, 0.0, 0.5}
    \definecolor{outcolor}{rgb}{0.545, 0.0, 0.0}



    
    % Prevent overflowing lines due to hard-to-break entities
    \sloppy 
    % Setup hyperref package
    \hypersetup{
      breaklinks=true,  % so long urls are correctly broken across lines
      colorlinks=true,
      urlcolor=urlcolor,
      linkcolor=linkcolor,
      citecolor=citecolor,
      }
    % Slightly bigger margins than the latex defaults
    
    \geometry{verbose,tmargin=1in,bmargin=1in,lmargin=1in,rmargin=1in}
    
    

    \begin{document}
    
    
    \maketitle
    
    

    
    \hypertarget{efectos-mixtos---ej5---ejercicios_mixtos.pdf}{%
\section{Efectos Mixtos - Ej5 -
ejercicios\_mixtos.pdf}\label{efectos-mixtos---ej5---ejercicios_mixtos.pdf}}

Mismo ejercicio pero para tratar como si fuesen:

\begin{itemize}
\tightlist
\item
  2 factores fijos
\item
  1 fijo y 1 aleatorio
\item
  2 aleatorios
\end{itemize}

Lectura de datos y conversión a factor:

    \begin{Verbatim}[commandchars=\\\{\}]
{\color{incolor}In [{\color{incolor}2}]:} datos \PY{o}{=} read.csv\PY{p}{(}\PY{l+s}{\PYZdq{}}\PY{l+s}{datos/ej5\PYZus{}ejercicios\PYZus{}mixtos.csv\PYZdq{}}\PY{p}{,} sep\PY{o}{=}\PY{l+s}{\PYZdq{}}\PY{l+s}{;\PYZdq{}}\PY{p}{)}
        str\PY{p}{(}datos\PY{p}{)}
        \PY{k+kn}{attach}\PY{p}{(}datos\PY{p}{)}
        velocidadf \PY{o}{=} \PY{k+kp}{as.factor}\PY{p}{(}velocidad\PY{p}{)}
        alimentacionf \PY{o}{=} \PY{k+kp}{as.factor}\PY{p}{(}alimentacion\PY{p}{)}
\end{Verbatim}


    \begin{Verbatim}[commandchars=\\\{\}]
'data.frame':	16 obs. of  3 variables:
 \$ fuerza      : num  2.7 2.78 2.83 2.86 2.45 2.49 2.85 2.8 2.6 2.72 {\ldots}
 \$ velocidad   : int  125 125 200 200 125 125 200 200 125 125 {\ldots}
 \$ alimentacion: num  0.015 0.015 0.015 0.015 0.03 0.03 0.03 0.03 0.045 0.045 {\ldots}

    \end{Verbatim}

    \hypertarget{modelos}{%
\subsection{Modelos}\label{modelos}}

\hypertarget{modelo-2-efectos-fijos}{%
\subsubsection{Modelo: 2 efectos fijos}\label{modelo-2-efectos-fijos}}

\begin{itemize}
\tightlist
\item
  \(\alpha\) (i): velocidad
\item
  \(\beta\) (j): alimentacion
\end{itemize}

\[y_{ijk} = \mu + \alpha_i + \beta_j + (\alpha \beta)_{ij} + \epsilon_{ijk}\]

    \textbf{Gráficos exploratorios}

    \begin{Verbatim}[commandchars=\\\{\}]
{\color{incolor}In [{\color{incolor}3}]:} interaction.plot\PY{p}{(}velocidadf\PY{p}{,} alimentacionf\PY{p}{,} fuerza\PY{p}{)}
\end{Verbatim}


    \begin{center}
    \adjustimage{max size={0.9\linewidth}{0.9\paperheight}}{output_4_0.png}
    \end{center}
    { \hspace*{\fill} \\}
    
    \begin{Verbatim}[commandchars=\\\{\}]
{\color{incolor}In [{\color{incolor}4}]:} interaction.plot\PY{p}{(}alimentacionf\PY{p}{,} velocidadf\PY{p}{,} fuerza\PY{p}{)}
\end{Verbatim}


    \begin{center}
    \adjustimage{max size={0.9\linewidth}{0.9\paperheight}}{output_5_0.png}
    \end{center}
    { \hspace*{\fill} \\}
    
    De los gráficos podemos inferir que la existe una interacción pero no
parece ser muy significativa (interacción leve).

A continuación hacemos la tabla anova para analizar numéricamente si son
significativos los factores y su combinación:

    \begin{Verbatim}[commandchars=\\\{\}]
{\color{incolor}In [{\color{incolor}5}]:} m1 \PY{o}{=} aov\PY{p}{(}fuerza\PY{o}{\PYZti{}}velocidadf\PY{o}{*}alimentacionf\PY{p}{)}
        \PY{k+kp}{summary}\PY{p}{(}m1\PY{p}{)}
\end{Verbatim}


    
    \begin{verbatim}
                         Df  Sum Sq Mean Sq F value   Pr(>F)    
velocidadf                1 0.14823 0.14823  57.010 6.61e-05 ***
alimentacionf             3 0.09250 0.03083  11.859  0.00258 ** 
velocidadf:alimentacionf  3 0.04187 0.01396   5.369  0.02557 *  
Residuals                 8 0.02080 0.00260                     
---
Signif. codes:  0 ‘***’ 0.001 ‘**’ 0.01 ‘*’ 0.05 ‘.’ 0.1 ‘ ’ 1
    \end{verbatim}

    
    Vemos que son significativas tanto la interacción de la velocidad con la
alimentación como cada factor individualmente.

Ahora veamos si se cumplen los \textbf{supuestos de normalidad y
igualdad de varianzas}.

    \begin{Verbatim}[commandchars=\\\{\}]
{\color{incolor}In [{\color{incolor}25}]:} qqnorm\PY{p}{(}m1\PY{o}{\PYZdl{}}residuals\PY{p}{)}
         qqline\PY{p}{(}m1\PY{o}{\PYZdl{}}residuals\PY{p}{)}
\end{Verbatim}


    \begin{center}
    \adjustimage{max size={0.9\linewidth}{0.9\paperheight}}{output_9_0.png}
    \end{center}
    { \hspace*{\fill} \\}
    
    \begin{Verbatim}[commandchars=\\\{\}]
{\color{incolor}In [{\color{incolor}6}]:} shapiro.test\PY{p}{(}m1\PY{o}{\PYZdl{}}residuals\PY{p}{)}
\end{Verbatim}


    
    \begin{verbatim}

	Shapiro-Wilk normality test

data:  m1$residuals
W = 0.96815, p-value = 0.8078

    \end{verbatim}

    
    \begin{Verbatim}[commandchars=\\\{\}]
{\color{incolor}In [{\color{incolor}7}]:} \PY{k+kn}{library}\PY{p}{(}car\PY{p}{)}
        leveneTest\PY{p}{(}m1\PY{p}{)}
\end{Verbatim}


    \begin{Verbatim}[commandchars=\\\{\}]
Loading required package: carData
Warning message in anova.lm(lm(resp \textasciitilde{} group)):
“ANOVA F-tests on an essentially perfect fit are unreliable”
    \end{Verbatim}

    \begin{tabular}{r|lll}
  & Df & F value & Pr(>F)\\
\hline
	group & 7             & 1.030707e+28  & 3.546276e-111\\
	  & 8             &           NA  &            NA\\
\end{tabular}


    
    Vemos que el test de levene no es confiable debido a que son muy pocas
observaciones, aún así debemos elegir si intentar hacer una
transformación para ver si mejora. Analizamos el gráfico de BoxCox,
corroborando si el 1 se encuentra en la banda de confianza.

    \begin{Verbatim}[commandchars=\\\{\}]
{\color{incolor}In [{\color{incolor}9}]:} \PY{k+kn}{library}\PY{p}{(}MASS\PY{p}{)}
        boxcox\PY{p}{(}m1\PY{p}{,} lambda \PY{o}{=} \PY{k+kp}{seq}\PY{p}{(}\PY{l+m}{\PYZhy{}10}\PY{p}{,} \PY{l+m}{20}\PY{p}{,} \PY{l+m}{1}\PY{o}{/}\PY{l+m}{10}\PY{p}{)}\PY{p}{)}
\end{Verbatim}


    \begin{center}
    \adjustimage{max size={0.9\linewidth}{0.9\paperheight}}{output_13_0.png}
    \end{center}
    { \hspace*{\fill} \\}
    
     ¿Por qué la banda de confianza nos da tan amplia, está bien esto? ¿La
transformación de BOXCOX se utiliza cuando al menos alguno de los
supuestos no se cumplen (varianza constante o normalidad)?

Como es significativa la interacción entre los factores fijos, no
podemos analizarlos en forma independiente, de aquí que no tiene sentido
ver un gráfico boxplot tomando los factores separados.

Debido a esta razón creamos un factor que sea la combinación de ambos
para luego sí hacer comparaciones múltiples usando el método de Tukey
que compara todos contra todos.

    \begin{Verbatim}[commandchars=\\\{\}]
{\color{incolor}In [{\color{incolor}11}]:} fav \PY{o}{=} \PY{k+kp}{factor}\PY{p}{(}\PY{k+kp}{paste0}\PY{p}{(}alimentacionf\PY{p}{,}velocidadf\PY{p}{)}\PY{p}{)}
         \PY{k+kn}{library}\PY{p}{(}multcomp\PY{p}{)}
         modelo \PY{o}{=} aov\PY{p}{(}fuerza\PY{o}{\PYZti{}}fav\PY{p}{)}
         mc \PY{o}{=} glht\PY{p}{(}modelo\PY{p}{,} linfct \PY{o}{=} mcp\PY{p}{(}fav\PY{o}{=}\PY{l+s}{\PYZdq{}}\PY{l+s}{Tukey\PYZdq{}}\PY{p}{)}\PY{p}{)}
         plot\PY{p}{(}mc\PY{p}{)}
\end{Verbatim}


    \begin{center}
    \adjustimage{max size={0.9\linewidth}{0.9\paperheight}}{output_15_0.png}
    \end{center}
    { \hspace*{\fill} \\}
    
    \begin{Verbatim}[commandchars=\\\{\}]
{\color{incolor}In [{\color{incolor}12}]:} \PY{c+c1}{\PYZsh{} Esto para ver las comparaciones que hace ya que no se ve en la gráfica}
         \PY{k+kp}{print}\PY{p}{(}mc\PY{p}{)}
\end{Verbatim}


    \begin{Verbatim}[commandchars=\\\{\}]

	 General Linear Hypotheses

Multiple Comparisons of Means: Tukey Contrasts


Linear Hypotheses:
                         Estimate
0.015200 - 0.015125 == 0    0.105
0.03125 - 0.015125 == 0    -0.270
0.03200 - 0.015125 == 0     0.085
0.045125 - 0.015125 == 0   -0.080
0.045200 - 0.015125 == 0    0.125
0.06125 - 0.015125 == 0     0.065
0.06200 - 0.015125 == 0     0.170
0.03125 - 0.015200 == 0    -0.375
0.03200 - 0.015200 == 0    -0.020
0.045125 - 0.015200 == 0   -0.185
0.045200 - 0.015200 == 0    0.020
0.06125 - 0.015200 == 0    -0.040
0.06200 - 0.015200 == 0     0.065
0.03200 - 0.03125 == 0      0.355
0.045125 - 0.03125 == 0     0.190
0.045200 - 0.03125 == 0     0.395
0.06125 - 0.03125 == 0      0.335
0.06200 - 0.03125 == 0      0.440
0.045125 - 0.03200 == 0    -0.165
0.045200 - 0.03200 == 0     0.040
0.06125 - 0.03200 == 0     -0.020
0.06200 - 0.03200 == 0      0.085
0.045200 - 0.045125 == 0    0.205
0.06125 - 0.045125 == 0     0.145
0.06200 - 0.045125 == 0     0.250
0.06125 - 0.045200 == 0    -0.060
0.06200 - 0.045200 == 0     0.045
0.06200 - 0.06125 == 0      0.105


    \end{Verbatim}

    Como el factor \emph{rapidez de la alimentación} es cuantitativo podemos
realizar un análisis de tendencia, del siguiente modo:

    \begin{Verbatim}[commandchars=\\\{\}]
{\color{incolor}In [{\color{incolor}27}]:} g \PY{o}{=} \PY{l+m}{4}
         contrasts\PY{p}{(}alimentacionf\PY{p}{)} \PY{o}{=} contr.poly\PY{p}{(}g\PY{p}{,} scores \PY{o}{=} \PY{k+kt}{c}\PY{p}{(}\PY{l+m}{0.015}\PY{p}{,} \PY{l+m}{0.030}\PY{p}{,} \PY{l+m}{0.045}\PY{p}{,} \PY{l+m}{0.060}\PY{p}{)}\PY{p}{)} 
         modelot \PY{o}{=} aov\PY{p}{(}fuerza\PY{o}{\PYZti{}}velocidadf\PY{o}{*}alimentacionf\PY{p}{)}
         summary.lm\PY{p}{(}modelot\PY{p}{)}
\end{Verbatim}


    
    \begin{verbatim}

Call:
aov(formula = fuerza ~ velocidadf * alimentacionf)

Residuals:
     Min       1Q   Median       3Q      Max 
-0.06000 -0.02625  0.00000  0.02625  0.06000 

Coefficients:
                              Estimate Std. Error t value Pr(>|t|)    
(Intercept)                    2.66875    0.01803 148.036 4.85e-15 ***
velocidadf200                  0.19250    0.02550   7.550 6.61e-05 ***
alimentacionf.L                0.08609    0.03606   2.388 0.044016 *  
alimentacionf.Q                0.20750    0.03606   5.755 0.000427 ***
alimentacionf.C               -0.11292    0.03606  -3.132 0.013976 *  
velocidadf200:alimentacionf.L -0.03354    0.05099  -0.658 0.529140    
velocidadf200:alimentacionf.Q -0.17500    0.05099  -3.432 0.008928 ** 
velocidadf200:alimentacionf.C  0.10062    0.05099   1.973 0.083904 .  
---
Signif. codes:  0 ‘***’ 0.001 ‘**’ 0.01 ‘*’ 0.05 ‘.’ 0.1 ‘ ’ 1

Residual standard error: 0.05099 on 8 degrees of freedom
Multiple R-squared:  0.9314,	Adjusted R-squared:  0.8715 
F-statistic: 15.53 on 7 and 8 DF,  p-value: 0.0004502

    \end{verbatim}

    
    Calculamos un \textbf{nuevo \(\alpha\)} para comparar, usando Bonferroni
\(\alpha_{PC} = 1 - (1- \alpha_{0QueQuiero})^{1/7}\), donde
\(\alpha_{0QueQuiero} = 0.05\). Según este nuevo \(\alpha\) veremos las
relaciones donde \(pvalor<0.0073\).

    \begin{Verbatim}[commandchars=\\\{\}]
{\color{incolor}In [{\color{incolor}31}]:} cant\PYZus{}comp \PY{o}{=} \PY{l+m}{7}
         alpha \PY{o}{=} \PY{l+m}{0.05}
         \PY{p}{(}alpha\PYZus{}PC \PY{o}{=} \PY{l+m}{1} \PY{o}{\PYZhy{}} \PY{p}{(}\PY{l+m}{1}\PY{o}{\PYZhy{}} alpha\PY{p}{)}\PY{o}{\PYZca{}}\PY{p}{(}\PY{l+m}{1}\PY{o}{/}cant\PYZus{}comp\PY{p}{)}\PY{p}{)}
\end{Verbatim}


    0.00730083197901465

    
     Vemos que el único pvalor inferior al nuevo \(\alpha\) es el
correspondiente a alimentacionf.Q, ¿Esta bien esto? ¿por qué no nos
aparece ningún pvalor inferior al nuevo \(\alpha\) en las interacciones?

    \hypertarget{modelo-1-efecto-fijo-y-1-aleatorio}{%
\subsubsection{Modelo: 1 efecto fijo y 1
aleatorio}\label{modelo-1-efecto-fijo-y-1-aleatorio}}

\hypertarget{velocidad-fija-y-alimentaciuxf3n-aleatorio}{%
\paragraph{Velocidad (fija) y alimentación
(aleatorio)}\label{velocidad-fija-y-alimentaciuxf3n-aleatorio}}

\begin{itemize}
\tightlist
\item
  \(\alpha\) (i): velocidad
\item
  \(B\) (j): alimentacion
\end{itemize}

\[y_{ijk} = \mu + \alpha_i + B_j + (\alpha B)_{ij} + \epsilon_{ijk}\]

Veamos un gráfico exploratorio boxplot:

    \begin{Verbatim}[commandchars=\\\{\}]
{\color{incolor}In [{\color{incolor}20}]:} boxplot\PY{p}{(}fuerza\PY{o}{\PYZti{}}velocidadf\PY{p}{)}
\end{Verbatim}


    \begin{center}
    \adjustimage{max size={0.9\linewidth}{0.9\paperheight}}{output_23_0.png}
    \end{center}
    { \hspace*{\fill} \\}
    
    \begin{Verbatim}[commandchars=\\\{\}]
{\color{incolor}In [{\color{incolor}13}]:} \PY{k+kn}{library}\PY{p}{(}lme4\PY{p}{)}
         m2 \PY{o}{=} lmer\PY{p}{(}fuerza\PY{o}{\PYZti{}}velocidadf \PY{o}{+} \PY{p}{(}\PY{l+m}{1}\PY{o}{|}alimentacionf\PY{p}{)} 
                   \PY{o}{+} \PY{p}{(}\PY{l+m}{1}\PY{o}{|}velocidadf\PY{o}{:}alimentacionf\PY{p}{)}\PY{p}{)}
         \PY{k+kp}{summary}\PY{p}{(}m2\PY{p}{)}
         confint\PY{p}{(}m2\PY{p}{)}
\end{Verbatim}


    \begin{Verbatim}[commandchars=\\\{\}]
Loading required package: Matrix

    \end{Verbatim}

    
    \begin{verbatim}
Linear mixed model fit by REML ['lmerMod']
Formula: 
fuerza ~ velocidadf + (1 | alimentacionf) + (1 | velocidadf:alimentacionf)

REML criterion at convergence: -27

Scaled residuals: 
     Min       1Q   Median       3Q      Max 
-1.20366 -0.58317 -0.08913  0.58432  1.39143 

Random effects:
 Groups                   Name        Variance Std.Dev.
 velocidadf:alimentacionf (Intercept) 0.005679 0.07536 
 alimentacionf            (Intercept) 0.004219 0.06495 
 Residual                             0.002600 0.05099 
Number of obs: 16, groups:  velocidadf:alimentacionf, 8; alimentacionf, 4

Fixed effects:
              Estimate Std. Error t value
(Intercept)    2.66875    0.05291  50.439
velocidadf200  0.19250    0.05907   3.259

Correlation of Fixed Effects:
            (Intr)
velocddf200 -0.558
    \end{verbatim}

    
    \begin{Verbatim}[commandchars=\\\{\}]
Computing profile confidence intervals {\ldots}

    \end{Verbatim}

    \begin{tabular}{r|ll}
  & 2.5 \% & 97.5 \%\\
\hline
	.sig01 & 0.00000000 & 0.15118683\\
	.sig02 & 0.00000000 & 0.17951358\\
	.sigma & 0.03345319 & 0.09151009\\
	(Intercept) & 2.56318754 & 2.77431245\\
	velocidadf200 & 0.06343907 & 0.32155936\\
\end{tabular}


    
    Vemos que tanto la interacción de velocidad y alimentación como la
alimentación no tienen efecto significativo ya que los IC tienen el cero
en su límite inferior. De esto modo ahora corroboraremos si el factor
fijo tiene efecto significativo comparando con un test el modelo
completo contra uno reducido.

    \begin{Verbatim}[commandchars=\\\{\}]
{\color{incolor}In [{\color{incolor}14}]:} mcompleto \PY{o}{=} lmer\PY{p}{(}fuerza\PY{o}{\PYZti{}}velocidadf \PY{o}{+} \PY{p}{(}\PY{l+m}{1}\PY{o}{|}alimentacionf\PY{p}{)} 
                          \PY{o}{+} \PY{p}{(}\PY{l+m}{1}\PY{o}{|}velocidadf\PY{o}{:}alimentacionf\PY{p}{)}\PY{p}{,} REML \PY{o}{=} \PY{k+kc}{FALSE}\PY{p}{)}
         mreducido \PY{o}{=} lmer\PY{p}{(}fuerza\PY{o}{\PYZti{}}\PY{l+m}{1} \PY{o}{+} \PY{p}{(}\PY{l+m}{1}\PY{o}{|}alimentacionf\PY{p}{)} 
                          \PY{o}{+} \PY{p}{(}\PY{l+m}{1}\PY{o}{|}velocidadf\PY{o}{:}alimentacionf\PY{p}{)}\PY{p}{,} REML \PY{o}{=} \PY{k+kc}{FALSE}\PY{p}{)}
         anova\PY{p}{(}mcompleto\PY{p}{,} mreducido\PY{p}{)}
\end{Verbatim}


    \begin{tabular}{r|llllllll}
  & Df & AIC & BIC & logLik & deviance & Chisq & Chi Df & Pr(>Chisq)\\
\hline
	mreducido & 4          & -20.95723  & -17.86688  & 14.47862   & -28.95723  &       NA   & NA         &         NA\\
	mcompleto & 5          & -25.51672  & -21.65378  & 17.75836   & -35.51672  & 6.559492   &  1         & 0.01043262\\
\end{tabular}


    
    Como vemos un pvalor es chico (\textless{}0.05) podemos rechazar la H0,
pero, el test anova usa una distribución chi2 para construir el pvalor
que podría darnos mal porque es aproximada, esto es bueno cuando tenemos
un n grande, sino no es muy confiable. ¿Puedo hacer algo mejor? usamos
esta otra lib

    \begin{Verbatim}[commandchars=\\\{\}]
{\color{incolor}In [{\color{incolor}17}]:} \PY{k+kn}{library}\PY{p}{(}pbkrtest\PY{p}{)}
         KRmodcomp\PY{p}{(}mcompleto\PY{p}{,}mreducido\PY{p}{)}
\end{Verbatim}


    
    \begin{verbatim}
F-test with Kenward-Roger approximation; computing time: 0.06 sec.
large : fuerza ~ velocidadf + (1 | alimentacionf) + (1 | velocidadf:alimentacionf)
small : fuerza ~ 1 + (1 | alimentacionf) + (1 | velocidadf:alimentacionf)
        stat    ndf    ddf F.scaling p.value  
Ftest 10.619  1.000  3.000         1 0.04718 *
---
Signif. codes:  0 ‘***’ 0.001 ‘**’ 0.01 ‘*’ 0.05 ‘.’ 0.1 ‘ ’ 1
    \end{verbatim}

    
    Vemos que con este método volvemos a obtener un pvalor menor a la
significancia por lo que no contradice el test anova previo, así que
corroboramos que ambos modelos no son equivalentes por lo concluímos que
el efecto fijo de la velocidad es significativo.

    \begin{Verbatim}[commandchars=\\\{\}]
{\color{incolor}In [{\color{incolor}19}]:} mc2 \PY{o}{=} glht\PY{p}{(}mcompleto\PY{p}{,} linfct\PY{o}{=}mcp\PY{p}{(}velocidadf\PY{o}{=}\PY{l+s}{\PYZdq{}}\PY{l+s}{Tukey\PYZdq{}}\PY{p}{)}\PY{p}{)}
         plot\PY{p}{(}mc2\PY{p}{)}
\end{Verbatim}


    \begin{center}
    \adjustimage{max size={0.9\linewidth}{0.9\paperheight}}{output_30_0.png}
    \end{center}
    { \hspace*{\fill} \\}
    
    Podemos concluir que no son estadísticamente equivalentes los dos tipos
de velocidades. Veamos qué sucede teniendo en cuenta el siguiente
modelo, que intercambia el factor fijo con el aleatorio:

\hypertarget{alimentaciuxf3n-fija-y-velocidad-aleatorio}{%
\paragraph{Alimentación (fija) y velocidad
(aleatorio)}\label{alimentaciuxf3n-fija-y-velocidad-aleatorio}}

\begin{itemize}
\tightlist
\item
  \(A\) (i): velocidad
\item
  \(\beta\) (j): alimentacion
\end{itemize}

\[y_{ijk} = \mu + A_i + \beta_j + (A \beta)_{ij} + \epsilon_{ijk}\]

    \begin{Verbatim}[commandchars=\\\{\}]
{\color{incolor}In [{\color{incolor}21}]:} m3 \PY{o}{=} lmer\PY{p}{(}fuerza\PY{o}{\PYZti{}}alimentacionf \PY{o}{+} \PY{p}{(}\PY{l+m}{1}\PY{o}{|}velocidadf\PY{p}{)} 
                   \PY{o}{+} \PY{p}{(}\PY{l+m}{1}\PY{o}{|}velocidadf\PY{o}{:}alimentacionf\PY{p}{)}\PY{p}{)}
         \PY{k+kp}{summary}\PY{p}{(}m3\PY{p}{)}
         confint\PY{p}{(}m3\PY{p}{)}
\end{Verbatim}


    
    \begin{verbatim}
Linear mixed model fit by REML ['lmerMod']
Formula: 
fuerza ~ alimentacionf + (1 | velocidadf) + (1 | velocidadf:alimentacionf)

REML criterion at convergence: -22.7

Scaled residuals: 
    Min      1Q  Median      3Q     Max 
-1.2326 -0.6721  0.0101  0.5513  1.2053 

Random effects:
 Groups                   Name        Variance Std.Dev.
 velocidadf:alimentacionf (Intercept) 0.005679 0.07536 
 velocidadf               (Intercept) 0.016783 0.12955 
 Residual                             0.002600 0.05099 
Number of obs: 16, groups:  velocidadf:alimentacionf, 8; velocidadf, 2

Fixed effects:
                   Estimate Std. Error t value
(Intercept)         2.79250    0.10900  25.619
alimentacionf0.03  -0.14500    0.08354  -1.736
alimentacionf0.045 -0.03000    0.08354  -0.359
alimentacionf0.06   0.06500    0.08354   0.778

Correlation of Fixed Effects:
            (Intr) al0.03 a0.045
almntcn0.03 -0.383              
almntc0.045 -0.383  0.500       
almntcn0.06 -0.383  0.500  0.500
    \end{verbatim}

    
    \begin{Verbatim}[commandchars=\\\{\}]
Computing profile confidence intervals {\ldots}

    \end{Verbatim}

    \begin{tabular}{r|ll}
  & 2.5 \% & 97.5 \%\\
\hline
	.sig01 &  0.00000000  &  0.112976825\\
	.sig02 &  0.02794357  &  0.401805671\\
	.sigma &  0.03345351  &  0.090804080\\
	(Intercept) &  2.55536607  &  3.029633995\\
	alimentacionf0.03 & -0.28203922  & -0.007960855\\
	alimentacionf0.045 & -0.16703922  &  0.107039145\\
	alimentacionf0.06 & -0.07203922  &  0.202039145\\
\end{tabular}


    
    Con este modelo vemos nuevamente que la interacción entre la velocidad y
la alimentación no es significativa ya que incluye el 0 en el IC, no así
la velocidad. Hacemos ahora la comparación entre este modelo completo y
sin tener en cuenta el factor fijo:

    \begin{Verbatim}[commandchars=\\\{\}]
{\color{incolor}In [{\color{incolor}22}]:} mcompleto2 \PY{o}{=} lmer\PY{p}{(}fuerza\PY{o}{\PYZti{}}alimentacionf \PY{o}{+} \PY{p}{(}\PY{l+m}{1}\PY{o}{|}velocidadf\PY{p}{)} 
                           \PY{o}{+} \PY{p}{(}\PY{l+m}{1}\PY{o}{|}velocidadf\PY{o}{:}alimentacionf\PY{p}{)}\PY{p}{,} REML\PY{o}{=}\PY{k+kc}{FALSE}\PY{p}{)}
         mreducido2 \PY{o}{=} lmer\PY{p}{(}fuerza\PY{o}{\PYZti{}}\PY{l+m}{1} \PY{o}{+} \PY{p}{(}\PY{l+m}{1}\PY{o}{|}velocidadf\PY{p}{)} 
                           \PY{o}{+} \PY{p}{(}\PY{l+m}{1}\PY{o}{|}velocidadf\PY{o}{:}alimentacionf\PY{p}{)}\PY{p}{,} REML\PY{o}{=}\PY{k+kc}{FALSE}\PY{p}{)}
         anova\PY{p}{(}mcompleto2\PY{p}{,} mreducido2\PY{p}{)}
\end{Verbatim}


    \begin{tabular}{r|llllllll}
  & Df & AIC & BIC & logLik & deviance & Chisq & Chi Df & Pr(>Chisq)\\
\hline
	mreducido2 & 4          & -22.20954  & -19.11919  & 15.10477   & -30.20954  &       NA   & NA         &         NA\\
	mcompleto2 & 7          & -23.20521  & -17.79709  & 18.60261   & -37.20521  & 6.995672   &  3         & 0.07203583\\
\end{tabular}


    
    \begin{Verbatim}[commandchars=\\\{\}]
{\color{incolor}In [{\color{incolor}23}]:} \PY{k+kn}{library}\PY{p}{(}pbkrtest\PY{p}{)}
         KRmodcomp\PY{p}{(}mcompleto2\PY{p}{,}mreducido2\PY{p}{)}
\end{Verbatim}


    
    \begin{verbatim}
F-test with Kenward-Roger approximation; computing time: 0.07 sec.
large : fuerza ~ alimentacionf + (1 | velocidadf) + (1 | velocidadf:alimentacionf)
small : fuerza ~ 1 + (1 | velocidadf) + (1 | velocidadf:alimentacionf)
       stat   ndf   ddf F.scaling p.value
Ftest 2.209 3.000 3.000         1   0.266
    \end{verbatim}

    
    Vemos tanto en el test de \texttt{anova} como en \texttt{KRmodcomp}
obtenemos un pvalor mayor que la significancia por lo que aceptamos la
hipótesis nula (H0), a partir de esto concluímos que el factor fijo
rapidez de alimentación no tiene efecto significativo.

    \hypertarget{modelo-2-aleatorios}{%
\subsubsection{Modelo: 2 aleatorios}\label{modelo-2-aleatorios}}

\begin{itemize}
\tightlist
\item
  \(A\) (i): velocidad
\item
  \(B\) (j): alimentacion
\end{itemize}

\[y_{ijk} = \mu + A_i + B_j + (A B)_{ij} + \epsilon_{ijk}\]

    \begin{Verbatim}[commandchars=\\\{\}]
{\color{incolor}In [{\color{incolor}24}]:} m4 \PY{o}{=} lmer\PY{p}{(}fuerza\PY{o}{\PYZti{}}\PY{l+m}{1} \PY{o}{+} \PY{p}{(}\PY{l+m}{1}\PY{o}{|}velocidadf\PY{p}{)} \PY{o}{+} \PY{p}{(}\PY{l+m}{1}\PY{o}{|}alimentacionf\PY{p}{)} 
                   \PY{o}{+} \PY{p}{(}\PY{l+m}{1}\PY{o}{|}velocidadf\PY{o}{:}alimentacionf\PY{p}{)}\PY{p}{)}
         \PY{k+kp}{summary}\PY{p}{(}m4\PY{p}{)}
         confint\PY{p}{(}m4\PY{p}{)}
\end{Verbatim}


    
    \begin{verbatim}
Linear mixed model fit by REML ['lmerMod']
Formula: 
fuerza ~ 1 + (1 | velocidadf) + (1 | alimentacionf) + (1 | velocidadf:alimentacionf)

REML criterion at convergence: -27.4

Scaled residuals: 
     Min       1Q   Median       3Q      Max 
-1.23677 -0.57463 -0.08913  0.61743  1.35832 

Random effects:
 Groups                   Name        Variance Std.Dev.
 velocidadf:alimentacionf (Intercept) 0.005679 0.07536 
 alimentacionf            (Intercept) 0.004219 0.06495 
 velocidadf               (Intercept) 0.016783 0.12955 
 Residual                             0.002600 0.05099 
Number of obs: 16, groups:  
velocidadf:alimentacionf, 8; alimentacionf, 4; velocidadf, 2

Fixed effects:
            Estimate Std. Error t value
(Intercept)   2.7650     0.1016   27.22
    \end{verbatim}

    
    \begin{Verbatim}[commandchars=\\\{\}]
Computing profile confidence intervals {\ldots}

    \end{Verbatim}

    \begin{tabular}{r|ll}
  & 2.5 \% & 97.5 \%\\
\hline
	.sig01 & 0.00000000 & 0.20209025\\
	.sig02 & 0.00000000 & 0.20942313\\
	.sig03 & 0.00000000 & 0.41893756\\
	.sigma & 0.03345319 & 0.09151536\\
	(Intercept) & 2.51874193 & 3.01125937\\
\end{tabular}


    
    De este resultado vemos que ninguno de los efectos aleatorios es
significativo, ya que todos tienen al 0 en el límite inferior del IC.


    % Add a bibliography block to the postdoc
    
    
    
    \end{document}
